\documentclass[a4paper]{article}
\usepackage[english]{babel}
\usepackage[utf8]{inputenc}

% mathermatics
\usepackage{amssymb} % useful math symbols
\usepackage{amsmath} % more useful math

% graphics
\usepackage{graphicx}
\usepackage{float}    % for more accurate graphics placement
\usepackage{fancyhdr} % for top header

% references
\usepackage{hyperref} % needed by cleveref, also provides clickable links
\usepackage{cleveref} % needed by autonum
\usepackage{autonum} % only add numbers to referenced equations

% formatting
\usepackage{enumitem} % provides easy change of labels in enumerate environment
\usepackage[top=3cm, bottom=4cm, width=17cm]{geometry} % for smaller page margins

% colors
\usepackage{xcolor}

% coding
% 
% uncomment the following after you have completed the required installation
% see https://github.com/gpoore/minted for info

%\usepackage{minted} 
%\definecolor{codeBgColor}{RGB}{240,240,240}




% very simple alias, \ex{} becomes the same as \subsubsection*{}
% TIP: remove the * in the line below if you want it numbered
\newcommand{\ex}[1]{\subsubsection*{#1}}




%Begining of the document
\begin{document}

\pagestyle{fancy} % use pagestyle with simple header (from fancyhdr)

%\pagenumbering{gobble} % uncomment to remove pagenumbering (in case of single page document)
\fancyhead[L]{Basic writing 101}
\fancyhead[C]{\textbf{Homework 3}}
\fancyhead[R]{John Johnson ID12345}


\ex{Exercise 3-11}

\begin{enumerate}[label=\alph*)]
  \item You can nest enumerate environments like below
  \item
    \begin{enumerate}[start = 2, label=\arabic*)]
      \item Since we're using the \texttt{enumitem} package, the start of an
        enumeration can easily be changed.
        \setcounter{enumii}{4} % enumii for first nested counter, enumi for root counter
      \item and points can be skipped by using \texttt{\\setcounter}
    \end{enumerate}
\end{enumerate}



\ex{Exercise 3-14}

\begin{enumerate}[label=\Alph*)] % google is your friend for label variations
  \item you can put most things into items, but when adding large sections you
    might want to use brackets to surround your item
%  \item{
%   LIKE 
%   THIS
%    }
\end{enumerate}



\ex{Exercise 3-21}

\begin{enumerate}
    \item{
            Code syntax highlight example. See the minted documentation on
            https://github.com/gpoore/minted for plugin install instructions.
            Syntax highlighting won't work without the correct python modules.

% UNCOMMENT THIS ONLY IF YOU HAVE PYGMENTIZE INSTALLED
%
%            % change c++ below to whatever language you want, supports most
%            \begin{minted}
%                [
%                    linenos,                % line numbers
%                    bgcolor=codeBgColor     % background color, remove for none
%                ]
%                {c++} 
%#include<iostream>
%
%using namespace std;
%
%int main() {
%    cout << "Look how nice code looks in latex" << endl;
%}
%            \end{minted}


    }

\end{enumerate}



\ex{That's it for now}

I don't expect anyone to use this example as there are a lot better example
documents out there. This is more of an example of how assignments with this 
simple template could turn out.



%% uncomment this if you need references. Edit the .bbl file with your references
%% and use "\cite{bibitem-label}" to cite
%\bibliography{template}

\end{document}

